\movetabledown=2in
\begin{rotatetable*}
\newcolumntype{z}{>{\raggedright\arraybackslash}p{1.75in}}
\begin{deluxetable*}{lcllrRrrDr@{$\pm$}llz}
\tablecaption{\normalsize Basic information for our sample of \sneia.
\label{tab:Targets}}
\tablewidth{0pt}
\tabletypesize{\footnotesize}
\tablehead{
\colhead{Target Name} & \colhead{Disc. Date} & \colhead{R.A.} & \colhead{Dec.} &  
\colhead{Obs.\tablenotemark{1}} & \colhead{$t_\text{first}$\tablenotemark{2}} & 
\colhead{$t_\text{last}$\tablenotemark{3}} & \colhead{$t_\text{next}$\tablenotemark{4}} & 
\multicolumn2c{Redshift} & \multicolumn2c{Distance} & \colhead{$A_V$\tablenotemark{5}} & \colhead{Reference(s)}
\vspace{-6pt} \\
\colhead{} & \colhead{} & \colhead{[h:m:s]} & \colhead{[d:m:s]} &
\colhead{} & \colhead{[days]} & 
\colhead{[days]} & \colhead{[days]} & 
\multicolumn2c{} & \multicolumn2c{[Mpc]} & \colhead{[mag]} & \colhead{}
}
\decimals
\startdata
===
\enddata
\tablenotetext{1}{Number of \galex observations in both bands}
\tablenotetext{2}{Number of days between the first \galex observation in either band and the discovery date}
\tablenotetext{3}{Number of days between the discovery date and the last \galex observation in either band}
\tablenotetext{4}{Number of days between the discovery date and the next \galex observation in either band}
\tablenotetext{5}{$V$-band galactic absorption}
\tablenotetext{a}{Offset between NED entry and \galex sky coordinates is larger than 30 kpc}
\tablenotetext{b}{Redshift-independent distance from \citet{Tully2016-Cosmicflows3}}
\tablecomments{The complete table will be available as supplementary material. A portion is shown here for guidance regarding its form and content.}
\end{deluxetable*}
\end{rotatetable*}